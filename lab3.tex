\documentclass{article}
\usepackage{indentfirst}
\author{Yan Nuoyuan}
\title{CS$2104$ Lab3 Report}
\begin{document}\large
\maketitle
\setlength{\parindent}{2em}
\noindent Q1.\\
\indent In this question, I write three relations. The first
relation is ordered witch will check whether the input list is ordered or not. The second relation is permute and I use permutation/3 function. This will generate all permutations of given list. Finally the sorted just generate all possible order of list and check witch one is ordered.\\
\bigskip

\noindent Q2.\\
\indent Firstly, I comment the ``append'' relation because in my system, their have already be a relation called append and it doesn't allowed me to redefine it. For work on your system, you can delete my comment. And I write three similar relations dutchred, dutchwhite and dutchblue, witch will extract different colors respectively. Morevoer, I use member/2 relation here to check if X is present in the list Y. And dutch relation will give you the correct order of these colors.\\
\bigskip

\noindent Q3.\\
\indent For mazepath1, it can not deal with cyclic graph. The method here is quite straight forward. Just check the next. But for mazepath2, I use an extra list to store the point I have passed already. So every time, I need check if the next point is in my extra list. If true, I will ignore this point.\\
\bigskip

\noindent Q4.\\
\indent For granny's age question, I calculate from Jane's age and look back upon.\\
\indent For the grocer question, I find that clpfd cannot deal with float. So I enlarge 100 times and the number is based on cent. The execution for this question is a little bit slowly, so you may need wait for a while.
\end{document}